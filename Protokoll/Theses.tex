%
%     Hauptdatei des Latex-Dokuments
%

%========================= Allgemeine Informationen =============================
% Angaben zum Autor, der Arbeit und dem Betreuer
\def \AUTHOR     {Viviane Bremer, Justus Dr�gem�ller, Manisch Kumar} 
\def \TITLE      {Fachlabor Mikromechatronik}
\def \TYPE       {Laborprotokoll}
\def \MATRIKELNR {4254652, 1234567, 1234567}
%\def \BETREUER   {Dipl.-Ing. Vor- und Zuname}
%==================== Ende allgemeine Informationen =============================



%
%		Header-Datei
%

\documentclass[
	pdftex,             % Ausgabe des Latex-Dokuments als PDF
	12pt,				        % Schriftgroesse 12pt
	a4paper,		      	% Layout fuer Din A4
	german,			      	% deutsche Sprache, global
	BCOR10mm,		        % DIV-Wert fuer die Erstellung des Satzspiegels
	DIV14,				      % Korrektur f�r die Bindung
	parskip=half,       % Absatzgr��e -> halber Zeilenabstand
	oneside,			      % Einseitiger Druck
	%twoside,		        % Zweiseitiger Druck
	openany,            % Kapitel k�nnen auf geraden und ungeraden Seiten beginnen
	numbers=noenddot,   % Kapitelnummern immer ohne Punkt
	bibliography=totoc,	% Literaturverzeichnis im Inhaltsverzeichnis angeben
	index=totoc			    % Index im Inhaltsverzeichnis angeben
]{scrreprt}


% PDF-Dokument formatieren
\usepackage[
	pdftitle={\TITLE},
	pdfauthor={\AUTHOR},
	pdfcreator={pdfLatex},
	pdfsubject={\TYPE\ am Institut f�r Mikrotechnik der Technischen Universit�t Braunschweig},
	pdfkeywords={\TYPE, Technische Universit�t Braunschweig, TU-BS, Institut f�r Mikrotechnik, IMT, \TITLE, \AUTHOR, \MATRIKELNR},
  colorlinks={true}, % Umrandung von Links nicht sichtbar
  linkcolor={black},
  citecolor={black},
  filecolor={black},
  urlcolor={black},
	pdflang={de},
	hyperindex=true
]{hyperref}

% Deutsche Rechtschreibung
\usepackage{german,ngerman}
\usepackage[latin1]{inputenc}
\usepackage[T1]{fontenc}
\usepackage{lmodern}		%Schriftart
\usepackage[all]{nowidow}		%Hurenkinder+Schusterjungen unterdr�cken

% Aktuelles Datum ermitteln
\usepackage[ngerman]{datenumber}

% Erweiterung f�r ein deutsches Literaturverzeichnis
\bibliographystyle{alphadin}

% Benutzerdefinierte Kopf- und Fu�zeile
\usepackage[automark]{scrpage2}
\pagestyle{scrheadings}
\setheadsepline{.4pt} % Linie unter dem Header

% Erweiterte Mathematikbibliotheken
\usepackage{amsmath}
\usepackage{amssymb}

% Ma�einhaeiten-Darstellung verbessern
\usepackage{units}

% Einbinden von externen PDF Dateien
\usepackage[final]{pdfpages}

% Zum einbinden von Grafiken  
\usepackage{graphicx}                         % Latex-Packages und Layout Konfiguration
\begin{document}                                % Beginn des Dokumentes
\pagenumbering{Roman}                           % Seitenzahlen in r�mischen Ziffern
\begin{titlepage}
\setlength{\voffset}{1.5cm}

\unitlength1mm
\begin{picture}(210,25)(31.38,25)
\put(0,0){\includegraphics[width=1.4\textwidth]{figures/header.jpg}} %[viewport=0 723 595 796]
\end{picture}

\center
\vspace{-4em}
\Large{\textsf{\TYPE}}
\vspace{3em}

\Huge{\textbf{\textsf{\TITLE}}}

\vspace*{\fill} 
\large{
	\textsf{
		angefertigt von\\
		\vspace{1em}
		\textbf{
			\AUTHOR \\
		}
		Matr.-Nr.: \MATRIKELNR\\
	}
}
\vspace{4em}
\large{
	\textsf{
		am\\
		\vspace{0.5em}
		\textbf{
			Institut f�r Mikrotechnik\\
			Technische Universit�t Braunschweig\\
		}
	}
}
\vspace{2em}
%\large{
%	\textsf{
%		Betreuer:\\
%		\vspace{0.5em}
%		\textbf{
%			\\
%			\\
%		}
%	}
%}
\vspace{7em}
\Large{
	\textbf{
		\setdatetoday
		\datemonthname \ \the\year
	}
}
\end{titlepage}
                    % Einbinden der Titelseite


%============================ optionale Seiten ==================================
% Seiten die vor dem Inhaltsverzeichnis erscheinen sollen

%\includepdf{chapters/Aufgabenstellung.pdf}     % Aufgabenstellung als pdf-Datei
%\addchap*{Eidesstattliche Erkl�rung}


\vspace*{5cm}


Hiermit erkl�re ich eidesstattlich, dass ich diese Arbeit eigenst�ndig angefertigt und keine anderen als die angegebenen Hilfsmittel verwendet habe.

\bigskip
Braunschweig den \today  % Eidesstattliche Erkl�rung (nur f�r Diplomarbeit erforderlich)

%========================== Ende optionale Seiten ===============================


\tableofcontents                                % Inhaltsverzeichnis
\newpage
\setcounter{page}{1}                            % Seitenzahl zur�cksetzten
\pagenumbering{arabic}                          % Seitenzahlen in arabischen Zahlen


%============================ Kapitel der Arbeit ==================================
% Hier sind die einzelnen Kapitel der Arbeit einzubinden

%\addchap{Symbolverzeichnis}
\label{Symbolverzeichnis}

\begin{tabbing}
\hspace*{3.5cm}                                  \=  \hspace*{10.0cm}                     \= \kill
\textbf{Formelzeichen}                           \> \textbf{Bedeutung}                    \> \textbf{Einheit} \\
~\\
${\bf I}$                                        \> Strom                                 \> A  \\
${\bf i}$                                        \> Getriebe�bersetzung                   \> -  \\
${\bf l}$                                        \> L�nge                                 \> m  \\
${\bf M}$                                        \> Drehmoment                            \> N  \\
${\bf m}$                                        \> Masse                                 \> Kg \\
${\bf n}$                                        \> Drehzahl                              \> 1/sec \\
${\bf R}$                                        \> Widerstand                            \> $\Omega$ \\
${\bf t}$                                        \> Zeit                                  \> sec \\
${\bf U}$                                        \> Spannung                              \> V \\
~\\
${\boldsymbol \alpha}$                           \> Drehwinkel von Gelenk 1               \> rad \\
${\boldsymbol \beta}$                            \> Drehwinkel von Gelenk 2               \> rad \\
${\boldsymbol \Delta \boldsymbol \varepsilon}$   \> Aufl�sung der Drehwinkelsensoren      \> rad/Imp. \\
${\boldsymbol \varepsilon}$                      \> Drehwinkel allgemein                  \> rad \\
${\boldsymbol \theta \textbf{, J}}$              \> Tr�gheitsmoment                       \> Kg$\cdot$m$^2$ \\
~\\
~\\
\textbf{Indizes}                                 \> \textbf{Bedeutung} \\
~\\
${\bf 0}$                                        \> Leerlauf \\
${\bf el}$                                       \> elektrischer Teil der Motoren \\
${\bf mech}$                                     \> mechanischer Teil der Motoren \\
${\bf G}$                                        \> Getriebe \\
${\textbf{i, j}}$                                \> Laufindex (1, 2, 3, ...) \\
${\bf Ref}$                                      \> Referenzwert \\
${\bf soll}$                                     \> Sollgr��e \\
\end{tabbing}

\chapter{Introduction}
\label{sec:Introduction}
Hier stehen einig einf�hrende Worte in das Thema der Arbeit.



\section{Zielsetzung und Gliederung}
\label{sec:Zielsetzung}
Was ist Sinn und Zweck der Arbeit, wie ist sie aufgebaut und welche Themen werden behandelt \cite{MechanikIII}.


\section{Geschichte/Literatur}
\label{sec:Geschichte_Literatur}
Optionale Betrachtung der Thematik wie sie in der Geschichte und/oder Literatur Erw�hnung findet. Hier sollten auch schon der eine \cite{MechanikI} oder andere \cite{MechanikII} Literaturverweise auftauchen. 

\chapter{Preliminary Design}
\label{sec:Preliminary Design}



\section{Wert und Einheit}
\label{sec:Unit}
Viele Einheiten lassen sich sch�ner darstellen mit dem "`Tag"' \verb|\unit[]{}| beziehungsweise \verb|\unitfrac[]{}{}|. Siehe den Vergleich: ohne 1~m oder mit \unit[1]{m} bzw. ohne 1~m/sec oder mit \unitfrac[1]{m}{sec}.

\section{�berschrift}
\label{sec:Quelltext}
Text Text Text Text Text Text Text Text Text Text Text Text Text Text Text Text Text Text Text Text Text Text Text Text Text Text Text Text Text Text Text Text Text Text Text Text Text Text Text Text Text Text Text Text Text Text Text Text Text Text Text Text Text Text Text Text Text Text Text Text Text Text Text Text Text Text Text Text Text Text Text Text

\section{Anforderungen an einen Drucksensor (Viviane Bremer)}
\label{sec:Anforderungen}

F�r die Auswertung des morphologischen Kastens muss zun�chst eine Anforderungsliste erstellt werden. Es soll ein Drucksensor zur �berwachung des Gasdrucks in einer Niederdruck-Pneumatikleitung entwickelt werden. Dieser Druck liegt gew�hnlich zwischen \unit[0]{} und \unit[1]{bar}, kann jedoch auf maximal \unit[1,2]{bar} ansteigen. Die Messgenauigkeit sollte hierbei mindestens bei \unit[$\pm 50$]{bar} liegen. Eine �nderung des Druckes verl�uft sehr langsam und kann somit als quasi-statisch angesehen werden. Da der Sensor in eine bestehendes Geh�use integriert wird, d�rfen seine Abma�e \unit[10x10]{mm} nicht �berschreiten. Das Ausgangssignal soll einer Ausgangsspannung von \unit[0]{} bis \unit[1]{V} entsprechen und somit den anliegenden Druck in \unit[]{bar} repr�sentieren. Zur Spannungsversorgen steht eine symmetrische Spannung von \unit[12]{V} und eine Referenzspannung von \unit[1]{V} zur Verf�gung. Des Weiteren wird mit einem Bedarf von \unit[2.000.000]{St�ck} gerechnet. Die Fertigungskosten sollten bezogen auf die St�ckzahl so gering wie m�glich ausfallen.

Diese Anforderungen sind nach Fest-, Mindest- und Wunschanforderung in Tabelle \textbf{blubb} aufgelistet.

\textbf{-->morphologischen Kasten bewerten}


\chapter{FEM-Simulation}
\label{sec:FEM-Simulation}



\section{Einfluss der Membrandicke auf die maximale Spannung (Viviane Bremer)}
\label{sec:Membrandicke}

In diesem Abschnitt wird der Einfluss der Membrandicke auf die auftretende mechanische Spannung in der Membran betrachtet. Hierf�r werden folgende Dicken genutzt: \unit[15]{$\mu$m}, \unit[25]{$\mu$m}, ,\unit[35]{$\mu$m}.

%\begin{itemize}
%	\item[a.] \unit[15]{$\mu m$}
%	\item[b.] \unit[25]{$\mu m$}
%	\item[c.] \unit[35]{$\mu m$}	
%\end{itemize}

Zur Analyse wird der anliegende Druck f�r alle drei Membrandicken in \unit[0.2]{bar} Schritten von \unit[0.2]{bar} bis \unit[1]{bar} erh�ht und die resultierende mechanische Spannung ermittelt. Die Verl�ufe sind in Abbildung \textbf{blubb} dargestellt. Die Bruchspannung $\sigma_{Br}$ von Silizium betr�gt \unit[830]{MPa} und sollte w�hrend des Sensorbetriebs nicht �berschritten werden. Zur besseren Visualisierung ist sie als Konstante im Graphen dargestellt.

Membran a erreicht die Bruchspannung schon bei einem Druck von \unit[0.439]{bar}, welcher im ben�tigten Messbereich liegt. Im Falle von Membran b wird sie knapp oberhalb des Messbereichs bei \unit[1.05]{bar} erreicht. Die Bruchspannung bei Membran c entspricht einem anliegenden Druck von \unit[2]{bar}, welcher deutlich oberhalb des Messbereiches ist. Hieraus lassen sich Erkenntnisse f�r das Sensorverhalten gewinnen. Membran a ist zu d�nn f�r diese Messaufgabe, da sie innerhalb des Messbereichs rei�t. Jedoch ist Membran c auch nicht akzeptabel, da niedrige Spannungen auftreten im Vergleich zu Membran b. Diese bildet einen guten Kompromiss f�r die gew�nschte Aufgabe.


\section{Einfluss der Membranabmessungen auf den Spannungsverlauf}
\label{sec:Membranabmessungen}

\subsection{Einfluss der Bossgr��e (Viviane Bremer)}
\label{subsec:Bosssize}

Zur Analyse des Einflusses der Bossgr��e wird die Membrangr��e auf \unit[3600]{$\mu$m} und die Membrandicke auf \unit[25]{$\mu$m} gesetzt. Die Bossgr��e wird folgenderma�en variiert: \unit[800]{$\mu$m}, \unit[1000]{$\mu$m}, \unit[1200]{$\mu$m}.

Die Spannungsverl�ufe und Verschiebungen sind in Abbildung \textbf{blubb} dargestellt. Bei einer Bossgr��e von \unit[800]{$\mu$m} treten Spannungen von bis zu \unit[274]{MPa} auf. Eine Vergr��erung des Bosses auf \unit[1000]{$\mu$m} f�hrt zu einer maximalen Spannung von  \unit[178]{MPa}. Der gr��te Boss senkt die auftretende Spannung auf \unit[128]{MPa}. Des Weiteren ist zu sehen, dass eine Vergr��erung des Bosses zu einer Angleichung der Spannungsspitzen f�hrt und die Dehnung der Membran von \unit[374]{$\mu$m} auf \unit[77]{$\mu$m} senkt.
\chapter{Etching Process}
\label{sec:EtchingProcess}


(Viviane Bremer)\\
F�r die Erstellung der Sensormembran muss zun�chst der �tzprozess charakterisiert werden. Zur Berechnung des zu �tzenden Volumens wird folgende Volumenformel eines Pyramidenstumpfes, 

\begin{equation}
	V = \frac{h}{3}(A_1 + \sqrt{A_1 A_2} + A_2), \label{eq:Pyramide}
\end{equation}
 

ben�tigt. F�r die Fertigung wird ein 4"-Wafer mit einer Dicke von \unit[450]{$\mu m$} genutzt. Die zu erzeugende Membran ohne Boss hat eine Breite von \unit[4000]{$\mu m$} und eine Dicke von \unit[25]{$\mu m$}. Dies f�hrt zu einer �tzh�he von

\begin{equation}
	h = 450\,\mu m - 25\,\mu m = 425\,\mu m.
\end{equation}


Die Fl�che $A_1$ bestimmt sich mit der Membranbreite zu

\begin{equation}
	A_1 = a_1^2 = 4000\,\mu m \cdot 4000\,\mu m = 16\,mm^2.
\end{equation}

Fl�che $A_2$ ergibt sich mit

\begin{align}
	\tan(54,7�) = \frac{425\,\mu m}{\Delta a} \\
	\Delta a = \frac{425\,\mu m}{\tan(54,7�)} = 300\,\mu m\\
	a_2 = a_1 - 2 \Delta a = 3400\,\mu m
\end{align}

zu

\begin{equation}
	A_2 = a_2^2 = 11,56\,mm^2.
\end{equation}

Dies eingesetzt in Gleichung \ref{eq:Pyramide} ergibt f�r das zu �tzende Volumen

\begin{equation}
	V_{Si} = \frac{0.425\,mm}{3}(16 + \sqrt{16 \cdot 11.56} +11.56)\,mm^2 = 5.831\,mm^3.
\end{equation}

Die Dichte $\rho_{Si}$ von Silizium betr�gt \unitfrac[0.002336]{g}{$mm^3$}. Dies f�hrt mit dem errechneten Volumen zu einer Masse von

\begin{equation}
	m_{Si} = \rho V_{Si} = 0.0137\,g
\end{equation}

Mit der Molmasse von Silizium, $M_{Si} = $\unitfrac[28.09]{g}{mol}, ergibt sich die Stoffmenge zu

\begin{equation}
	n_{Si} = \frac{m_{Si}}{M_{Si}} = 4.877 \cdot 10^{-4}.
\end{equation}




\chapter{Characterising}
\label{sec:Characterising}



\section{Wert und Einheit}
\label{sec:Unit}
Viele Einheiten lassen sich sch�ner darstellen mit dem "`Tag"' \verb|\unit[]{}| beziehungsweise \verb|\unitfrac[]{}{}|. Siehe den Vergleich: ohne 1~m oder mit \unit[1]{m} bzw. ohne 1~m/sec oder mit \unitfrac[1]{m}{sec}.

\section{�berschrift}
\label{sec:Quelltext}
Text Text Text Text Text Text Text Text Text Text Text Text Text Text Text Text Text Text Text Text Text Text Text Text Text Text Text Text Text Text Text Text Text Text Text Text Text Text Text Text Text Text Text Text Text Text Text Text Text Text Text Text Text Text Text Text Text Text Text Text Text Text Text Text Text Text Text Text Text Text Text Text

\section{Abbildungen einbinden}
\label{sec:pictures}
Text Text Text Text Text Text Text Text Text Text Text Text Text Text Text Text Text Text Text Text Text Text Text Text Text Text Text Text Text Text Text Text Text Text Text Text Text Text Text Text Text Text Text Text Text Text Text Text Text Text Text Text Text Text Text Text Text Text Text Text Text Text Text Text Text Text Text Text Text Text Text Text Text Text Text Text Text Text 


\begin{figure}[htbp]
  \centering
  \includegraphics{figures/empty.jpg}
  \caption{Einzelne Abbildung}
  \label{fig:singlepicture}
\end{figure}


Text Text Text Text Text Text Text Text Text Text Text Text Text Text Text Text Text Text Text Text Text Text Text Text Text Text Text Text Text Text Text Text Text Text Text Text Text Text Text Text Text Text Text Text Text Text Text Text Text Text Text Text Text Text Text Text Text Text Text Text Text Text Text Text Text Text Text Text Text Text Text Text Text Text Text Text Text Text Text Text Text Text Text Text Text Text Text Text Text Text Text Text Text Text Text Text Text Text Text Text Text Text Text Text Text Text Text Text Text Text Text Text Text Text Text Text Text Text Text Text Text Text Text Text Text Text Text Text Text Text Text Text Text Text Text Text Text Text Text Text Text Text Text Text Text Text Text Text Text 



Text Text Text Text Text Text Text Text Text Text Text Text Text Text Text Text Text Text Text Text Text Text Text Text Text Text Text Text Text Text Text Text Text Text Text Text Text Text Text Text Text Text Text Text Text Text Text Text Text Text Text Text Text Text Text Text Text Text Text Text Text Text Text Text Text Text Text Text Text Text Text Text Text Text Text Text Text Text Text Text Text Text Text Text Text Text Text Text Text Text Text Text Text Text Text Text Text Text Text Text Text Text Text Text Text Text Text Text Text Text Text Text Text Text Text Text Text Text Text Text Text Text Text Text Text Text Text Text Text Text Text Text Text 
\input{chapters/circuit-layout}
\input{chapters/evaluation-circuit}
\chapter{Zusammenfassung und Ausblick}
\label{sec:Zusammenfassung}
Hier stehen die Ergebnisse der Arbeit und ein kurzer Ausblick wie es weiter gehen kann.

%=============================== Ende Kapitel =====================================

\bibliography{chapters/bibliography}              % Literaturverzeichnis

% Anhang (Formatierung)
\appendix
\chapter*{Appendix}
\addcontentsline{toc}{chapter}{appendix}
\setcounter{chapter}{1}
\markboth{Anhang}{Anhang}
\label{Anhang}


%============================ Kapitel des Anhangs ==================================
% Kapitel die als Anhang angef�gt werden sollen

\section{Erster Anhang}
\label{sec:Anhang_1}
Ein Anhang.


\section{Zweiter Anhang}
\label{sec:Anhang_2}
Ein weiterer Anhang.



%================================ Ende Anhangs =====================================


\end{document}                                    % Ende des Dokumentes

