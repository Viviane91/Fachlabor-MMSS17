\chapter{Etching Process}
\label{sec:EtchingProcess}


(Viviane Bremer)\\
F�r die Erstellung der Sensormembran muss zun�chst der �tzprozess charakterisiert werden. Zur Berechnung des zu �tzenden Volumens wird folgende Volumenformel eines Pyramidenstumpfes, 

\begin{equation}
	V = \frac{h}{3}(A_1 + \sqrt{A_1 A_2} + A_2), \label{eq:Pyramide}
\end{equation}
 

ben�tigt. F�r die Fertigung wird ein 4"-Wafer mit einer Dicke von \unit[450]{$\mu m$} genutzt. Die zu erzeugende Membran ohne Boss hat eine Breite von \unit[4000]{$\mu m$} und eine Dicke von \unit[25]{$\mu m$}. Dies f�hrt zu einer �tzh�he von

\begin{equation}
	h = 450\,\mu m - 25\,\mu m = 425\,\mu m.
\end{equation}


Die Fl�che $A_1$ bestimmt sich mit der Membranbreite zu

\begin{equation}
	A_1 = a_1^2 = 4000\,\mu m \cdot 4000\,\mu m = 16\,mm^2.
\end{equation}

Fl�che $A_2$ ergibt sich mit

\begin{align}
	\tan(54,7�) = \frac{425\,\mu m}{\Delta a} \\
	\Delta a = \frac{425\,\mu m}{\tan(54,7�)} = 300\,\mu m\\
	a_2 = a_1 - 2 \Delta a = 3400\,\mu m
\end{align}

zu

\begin{equation}
	A_2 = a_2^2 = 11,56\,mm^2.
\end{equation}

Dies eingesetzt in Gleichung \ref{eq:Pyramide} ergibt f�r das zu �tzende Volumen

\begin{equation}
	V_{Si} = \frac{0.425\,mm}{3}(16 + \sqrt{16 \cdot 11.56} +11.56)\,mm^2 = 5.831\,mm^3.
\end{equation}

Die Dichte $\rho_{Si}$ von Silizium betr�gt \unitfrac[0.002336]{g}{$mm^3$}. Dies f�hrt mit dem errechneten Volumen zu einer Masse von

\begin{equation}
	m_{Si} = \rho V_{Si} = 0.0137\,g
\end{equation}

Mit der Molmasse von Silizium, $M_{Si} = $\unitfrac[28.09]{g}{mol}, ergibt sich die Stoffmenge zu

\begin{equation}
	n_{Si} = \frac{m_{Si}}{M_{Si}} = 4.877 \cdot 10^{-4}.
\end{equation}



