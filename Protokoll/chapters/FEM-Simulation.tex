\chapter{FEM-Simulation}
\label{sec:FEM-Simulation}



\section{Einfluss der Membrandicke auf die maximale Spannung (Viviane Bremer)}
\label{sec:Membrandicke}

In diesem Abschnitt wird der Einfluss der Membrandicke auf die auftretende mechanische Spannung in der Membran betrachtet. Hierf�r werden folgende Dicken genutzt: \unit[15]{$\mu$m}, \unit[25]{$\mu$m}, ,\unit[35]{$\mu$m}.

%\begin{itemize}
%	\item[a.] \unit[15]{$\mu m$}
%	\item[b.] \unit[25]{$\mu m$}
%	\item[c.] \unit[35]{$\mu m$}	
%\end{itemize}

Zur Analyse wird der anliegende Druck f�r alle drei Membrandicken in \unit[0.2]{bar} Schritten von \unit[0.2]{bar} bis \unit[1]{bar} erh�ht und die resultierende mechanische Spannung ermittelt. Die Verl�ufe sind in Abbildung \textbf{blubb} dargestellt. Die Bruchspannung $\sigma_{Br}$ von Silizium betr�gt \unit[830]{MPa} und sollte w�hrend des Sensorbetriebs nicht �berschritten werden. Zur besseren Visualisierung ist sie als Konstante im Graphen dargestellt.

Membran a erreicht die Bruchspannung schon bei einem Druck von \unit[0.439]{bar}, welcher im ben�tigten Messbereich liegt. Im Falle von Membran b wird sie knapp oberhalb des Messbereichs bei \unit[1.05]{bar} erreicht. Die Bruchspannung bei Membran c entspricht einem anliegenden Druck von \unit[2]{bar}, welcher deutlich oberhalb des Messbereiches ist. Hieraus lassen sich Erkenntnisse f�r das Sensorverhalten gewinnen. Membran a ist zu d�nn f�r diese Messaufgabe, da sie innerhalb des Messbereichs rei�t. Jedoch ist Membran c auch nicht akzeptabel, da niedrige Spannungen auftreten im Vergleich zu Membran b. Diese bildet einen guten Kompromiss f�r die gew�nschte Aufgabe.


\section{Einfluss der Membranabmessungen auf den Spannungsverlauf}
\label{sec:Membranabmessungen}

\subsection{Einfluss der Bossgr��e (Viviane Bremer)}
\label{subsec:Bosssize}

Zur Analyse des Einflusses der Bossgr��e wird die Membrangr��e auf \unit[3600]{$\mu$m} und die Membrandicke auf \unit[25]{$\mu$m} gesetzt. Die Bossgr��e wird folgenderma�en variiert: \unit[800]{$\mu$m}, \unit[1000]{$\mu$m}, \unit[1200]{$\mu$m}.

Die Spannungsverl�ufe und Verschiebungen sind in Abbildung \textbf{blubb} dargestellt. Bei einer Bossgr��e von \unit[800]{$\mu$m} treten Spannungen von bis zu \unit[274]{MPa} auf. Eine Vergr��erung des Bosses auf \unit[1000]{$\mu$m} f�hrt zu einer maximalen Spannung von  \unit[178]{MPa}. Der gr��te Boss senkt die auftretende Spannung auf \unit[128]{MPa}. Des Weiteren ist zu sehen, dass eine Vergr��erung des Bosses zu einer Angleichung der Spannungsspitzen f�hrt und die Dehnung der Membran von \unit[374]{$\mu$m} auf \unit[77]{$\mu$m} senkt.